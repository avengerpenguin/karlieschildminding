\documentclass[10pt,a4paper]{article}

\usepackage{fullpage}
\usepackage{tabularx}
\usepackage{colortbl}
\usepackage{xcolor}
\usepackage{fontspec}
\usepackage{xcolor}
\usepackage{titlesec}
\defaultfontfeatures{Ligatures=TeX}
% Set sans serif font to Calibri
\setsansfont{Carlito}
% Set serifed font to Cambria
\setmainfont{Caladea}
% Define light and dark Microsoft blue colours
\definecolor{MSBlue}{rgb}{.204,.353,.541}
\definecolor{MSLightBlue}{rgb}{.31,.506,.741}
% Define a new fontfamily for the subsubsection font
% Don't use \fontspec directly to change the font
\newfontfamily\subsubsectionfont[Color=MSLightBlue]{Times New Roman}
% Set formats for each heading level
\titleformat*{\section}{\huge\bfseries\sffamily\color{MSBlue}}
\titleformat*{\subsection}{\Large\bfseries\sffamily\color{MSLightBlue}}
\titleformat*{\subsubsection}{\itshape\subsubsectionfont}

\setcounter{secnumdepth}{0}

\setlength{\parindent}{0pt}
\setlength{\parskip}{2\baselineskip}%

\pagestyle{empty}

\begin{document}
\section{Safeguarding Policy}

\subsection{Child Protection}

The purpose of this policy is to provide protection for all children in
my care --- this is my first responsibility and they are my priority.
Outlined is the procedure I will adopt if I have any cause for concern,
including how -- if necessary -- I will make a referral to the
children's safeguarding team (Multi Agency Safeguarding Hub or MASH).

I understand that child abuse can be physical, emotional, sexual,
neglectful or a mixture of these. I must notify MASH and I am required
to record my concerns. I hold a copy of the relevant local procedures
and they are available on request.

Parents must notify me of any concerns they have about their child and
any accidents, incidents or injuries affecting the child, which will be
recorded.

If I notice:
\begin{itemize}
\item
  Significant changes in the child's behaviour;
\item
  Unexpected bruising, marks or signs of possible abuse;
\item
  Comments that give me cause for concern;
\item
  Deterioration in general wellbeing that cause concern; or
\item
  Signs of neglect.
\end{itemize}

I will implement the procedure detailed below without delay to minimise
any risk to the child.

The following procedure will be undertaken in the event that I suspect a
child is at risk of harm:

\begin{itemize}
\item
  Concerns will be discussed with parents, if appropriate.
\item
  Advice will be sought from MASH (0161 603 4500 -- or 0161 794 8888
  out of hours)
\item
  The ``Referral to Children's Services'' form will be completed
  (http://www.salford.gov.uk/secureupload.htm)
\item
  Once this referral is received, RIAT (Referral \& Initial Assessment
  Team) will take responsibility and decide the necessary steps to
  ensure the safety of the child.
\item
  In the event of an emergency, I will contact the police.
\end{itemize}

All information about the child is kept confidential, however in certain
situations, this information may be shared with MASH, RIAT or the
police.

\subsection{Allegations of Abuse}

In the event of an allegation being made against the childminder -- or
any other adult living in the childminder's home -- The Local Authority
Designated Officer (LADO) will be informed immediately. All allegations
will be investigated independently by the LADO and will also be reported
to Ofsted.

In all instances, I will record:

\begin{itemize}
\item
  The child's full name and address;
\item
  The date and time the record is being made;
\item
  Factual details of the concern, for example: bruising, what the
  child said and who was present;
\item
  Details of any previous concerns;
\item
  Details of any explanations from the parents; and
\item
  Any action taken, such as speaking to parents.
\end{itemize}

It is not my responsibility to attempt to investigate the situation
myself.

\subsection{Mobile Phones and Social Networking}

It is in the safety of the children in the childminder's care that the
childminder is in possession of a mobile phone at all times. This is to
be used to contact parents in the event of an emergency. The mobile
phone will not be used to take photographs of the child; these will all
be taken on the digital camera. This camera will be available for the
parent to view at any time, without appointment. For the safety of other
children in my care, I request that parents do not use their mobile
phones within the setting.

I will not post any photos or details of any children in my care on any
social network.

Date: 07/06/2014

\subsection{Useful Telephone Numbers}

\begin{table}[h]
  \begin{tabularx}{\textwidth}{lX}
    Ofsted & 08456 404040 \\
    LADO & 0161 603 4328 \\
    MASH & 0161 603 4500 (Out of hours 0161 794 8888) \\
    Starting Life Well & 0161 909 6508 \\ 
 \end{tabularx}
\end{table}

\begin{table}[h]
  \def\arraystretch{2.0}
  \begin{tabularx}{\textwidth}{|l|X|}
    \hline
    Childminder's name & \\
    \hline
    Childminder's signature &  \\
    \hline
    Date & \\
    \hline
  \end{tabularx}
\end{table}

\begin{table}[h]
  \def\arraystretch{2.0}
  \begin{tabularx}{\textwidth}{!{\color{gray}\vrule}X!{\color{gray}\vrule}X!{\color{gray}\vrule}X!{\color{gray}\vrule}X!{\color{gray}\vrule}}
    \arrayrulecolor{gray}\hline
    Child's Name & Parent's Name & Signature & Date \\
    \hline
    ~ & ~ & ~ & \\
    \hline
    ~ & ~ & ~ & \\
    \hline
    ~ & ~ & ~ & \\
    \hline
    ~ & ~ & ~ & \\
    \hline
    ~ & ~ & ~ & \\
    \hline
    ~ & ~ & ~ & \\
    \hline
  \end{tabularx}
\end{table}

\end{document}
