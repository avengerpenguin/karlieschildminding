\documentclass[10pt,a4paper]{report}

\usepackage[Bjarne]{fncychap}
%\usepackage{fullpage}
\usepackage{tabularx}
\usepackage{colortbl}
\usepackage{xcolor}
\usepackage{fontspec}
\usepackage{amsmath}
\usepackage{titlesec}
\defaultfontfeatures{Ligatures=TeX}
% Set sans serif font to Calibri
\setsansfont{Carlito}
% Set serifed font to Cambria
\setmainfont{Caladea}
% Define light and dark Microsoft blue colours
\definecolor{MSBlue}{rgb}{.204,.353,.541}
\definecolor{MSLightBlue}{rgb}{.31,.506,.741}
% Define a new fontfamily for the subsubsection font
% Don't use \fontspec directly to change the font
\newfontfamily\subsectionfont[Color=MSLightBlue]{Times New Roman}
% Set formats for each heading level
\titleformat{\chapter}[display]{\huge\bfseries\sffamily\color{MSBlue}}{}{0pt}{\huge}
\titleformat*{\section}{\Large\bfseries\sffamily\color{MSLightBlue}}
\titleformat*{\subsection}{\itshape\subsectionfont}

\usepackage{calc}
\makeatletter
\newcommand{\DESCRIPTION@original@item}{}
\let\DESCRIPTION@original@item\item
\newcommand*{\DESCRIPTION@envir}{DESCRIPTION}
\newlength{\DESCRIPTION@totalleftmargin}
\newlength{\DESCRIPTION@linewidth}
\newcommand{\DESCRIPTION@makelabel}[1]{\llap{#1}}%
\newcommand{\DESCRIPTION@item}[1][]{%
  \setlength{\@totalleftmargin}%
       {\DESCRIPTION@totalleftmargin+\widthof{\textbf{#1 }}-\leftmargin}%
  \setlength{\linewidth}
       {\DESCRIPTION@linewidth-\widthof{\textbf{#1 }}+\leftmargin}%
  \par\parshape \@ne \@totalleftmargin \linewidth
  \DESCRIPTION@original@item[\textbf{#1}]%
}
\newenvironment{DESCRIPTION}
  {\list{}{\setlength{\labelwidth}{0cm}%
           \let\makelabel\DESCRIPTION@makelabel}%
   \setlength{\DESCRIPTION@totalleftmargin}{\@totalleftmargin}%
   \setlength{\DESCRIPTION@linewidth}{\linewidth}%
   \renewcommand{\item}{\ifx\@currenvir\DESCRIPTION@envir
                           \expandafter\DESCRIPTION@item
                        \else
                           \expandafter\DESCRIPTION@original@item
                        \fi}}
  {\endlist}
  \makeatother
  
\title{\huge\bfseries\sffamily\color{MSBlue}Self-Evaluation Form}
\author{\Large\bfseries\sffamily\color{MSLightBlue}Karlie Vanes, Childminder}
\date{\itshape\subsectionfont January 2017 version}

\begin{document}

\maketitle

\tableofcontents

\chapter{My setting}

My setting is a warm family home. The most used rooms are the play room, the kitchen and the living room. The play room is where most of our free play and planned activities happen. There are  resources freely accessible and for young babies: toys are provided and rotated frequently so that they can experience a wide range of resources to entertain, challenge and stimulate their learning. The kitchen is used for our group dining and often for our messy activities, such as sensory resources like edible tapioca pearls, basil seed water play, home made edible slime etc. We also find ourselves singing and moving to music a lot in the kitchen, due to the wide open space and the young children's love of the area. The living room is available for nap times, quiet times and for any child who wishes to retreat from the noise of the play room and sit quietly. We have also used the living room space with older children to watch planned DVDs that suit a seasonal theme. We have a private, secure garden that is used for picnics, sand play, water play and messy play. The young children also enjoy sitting quietly outside listening out for birdsong and to the wind blowing through the trees. 

My home is on a residential street, 5 minutes walk from a large local green space and secure children's play area. This play area received a wonderful, funded improvement recently and as a result is a lovely place for children to climb, explore and socialise. 

My setting has two doors at the main entrance with a small step for each, these can be easily navigated with a pram, but wheelchair users would require assistance. The setting is open Monday- Friday 7:30am- 5:30 pm, but we currently only have children on register Tuesday, Wednesday, Thursday, term time only. I work alone, with no assistants or other Childminders and have no support staff. Maximum numbers are 2 early years children and 3 school-aged children aged five to eight. This is due to my having my own 4 year old counted in my ratios, at all times. We currently (as of 09/01/17) have one child on register who is 23 months old. She has no SEN or disabilities and doesn't use EAL or speak a language other than English. 

As the main provider, I attended Observation, Assessment and Planning training, as well as Enabling Environments training. Both of these gave me some brilliant insights into improved paper systems and interesting ideas for improving my resources and use of space, such as a good idea for a new weekly planning system, an improved feedback questionnaire and an inspired solution to an outdoor sand and water area that would survive harsh weather conditions. I am frequently on the look-out for new training opportunities as I love to improve my knowledge, awareness and learn things that can be beneficial to my setting. 

\section{Training Undertaken}

\begin{DESCRIPTION}
\item [07/12/13] First Aid for Childcare (The Training Co.) (12 hrs)
\item [05/01/14] EYFS Tracker Training (Starting Life Well) (2 hrs)
\item [09/01/14] Ofsted Inspector Framework Training (Starting Life Well) (2 hrs)
\item [04/02/14] Child Protection (Starting Life Well) (3 hrs)
\item [27/02/14] Two Year Progress Check Training (Starting Life Well) (2 hrs)
\item [27/03/14] Level 2 Food Safety \& Hygiene (Catering) (Virtual College) (6 hrs)
\item [13/01/16] Observation, Assessment \& Planning (artofchildcare training ltd) (3 hrs)
\item [10/02/16] Enabling Environments (artofchildcare training ltd) (3 hrs)
\item [08/03/16] Forced Marriage Training (e-learning Virtual College) (2 hrs)
\item [08/03/16] Domestic Abuse Basic Awareness Training (IDAS) (1 hr)
\item [09/03/16] Introduction to FGM, Forced Marriage, Spirit Possession and HBV (V2.1) (e-learning Virtual College) (1 hr 15 mins)
\item [09/03/16] CHANNEL PREVENT Training (e-learning College of Policing) (30 mins)
\item [09/03/16] Safeguarding Children Refresher (V2.1) (e-learning Virtual College) (3 hrs 50 mins)
\item [11/03/16] Early Child Development – Foundation (e-learning Virtual College) (2 hrs)
\item [18/03/16] Nutrition in the Early Years (Laser Learning) (1 hr)
\item [13/06/16] Safe Sleep Guidance Cluster Training (1hr 30 mins) (Starting Life Well)
\item [20/06/16] General Update Cluster Training (2 hrs) (Starting Life Well)
\item [07/11/16] General Update Cluster Training (2 hrs) (Starting Life Well)
\item [20/11/16] First Aid for Childcare (The Training Co.) (12 hrs)
\end{DESCRIPTION}

\chapter{Views of those who use my setting and who work with me}

\section{Working in partnership with parents}

I pride myself on working in partnership with parents and think that it is in the best interest of each child that we work together and frequently share information relating to the children's development and habits. I often share photos with parents of what their child gets up to, especially if I have noticed a new “wow” moment or a new pattern of behaviour. I share next steps and chat daily to parents about what their child has enjoyed, what I plan to do next in order to support their child's learning whilst enabling them to have fun. Mostly I approach parents how I would like to be approached. I have always been passionate about my child's development and feel that just because parent's are using childcare doesn't mean that they should miss out on all of the wonderful things that their child is learning and experiencing. Parents are encouraged to contribute to their child's Learning Journeys and to share any milestones achieved at home. Photos are passed back and forth via WhatsApp and comments are encouraged to be added in the child's daily diaries so that I can plan accordingly and work in a way that best suits the child as a unique individual. Information shared from parents to date includes a child's first steps, new methods of behaviour management, a child's first dental visit, new schemas that have been observed, a new climbing habit that merits constant vigilance etc.

\section{Feedback from parents}

I hand out questionnaires every term to parents in order to discover their level of satisfaction with the setting and if they have any ideas on how to improve my services. This feedback so far has been overwhelmingly positive and I have been asked for one addition by a parent, that I write down the times that I serve my meals in the daily diary. This was immediately implemented and has been the norm for this parent ever since. In February 2016 I extended my questionnaires to have a wider range of questions to improve the quality of feedback I received. 

Feedback has included comments such as: “Daily diaries: useful at bedtime knowing how much … has slept during day” “Girls love it here, very happy” “Daily diaries: I enjoy reading them. Information is really good” “I am over all very happy and satisfied with the care received” “Daily diaries: Happy, enjoy reading 'What I enjoyed today...'”
Parents have commented that they were unaware of the Early Years Foundation Stage (EYFS) until I pointed it out to them and explained it and that my end of term reports have helped to assist them when their child's “1 year check” was being completed. They have also commented that they have made multiple copies of these assessments and distributed them amongst their child's family so that they could all tailor their care in a way that would support the child's learning and help to move towards achieving their next steps. The next steps that I provide on observations have also been useful in the designing of gift lists for this child so that the parent knew which toys their child would benefit from and what would help them to progress through the pathway of the EYFS. Parents have also commented on their love of our going to a toddler group which encourages their child to socialise and to make craft items. Feedback has also been given on the frequent sharing of information that I pride myself on, such as activities a child has been doing and the resources that they enjoy.

\section{Working with schools}

I am not currently sharing care of any child with another setting, but in the past when I did have a child on register who also went to a nursery within a school, I requested a copy of the child's tracker to help with my starting points in their learning journey and frequently discussed their progress in person when I was collecting and dropping them off at the setting. In the future if I have to share care with another setting I have many plans that will work in the best interest of the child. I want to have a setting diary that stays in the child's bag and moves between settings for us to share information and any milestones achieved and also arrange meetings for us to compare Learning Journeys and discuss the child's development. I am aware that Childminders and school settings can experience childcare very differently and see different aspects to a child's learning, development, behaviour, play schemas etc. So a two way relationship like this would greatly benefit both settings.

\section{Feedback from children}

As the only child on register at the moment is 23 months old, (as of 09/01/17) their feedback and input is mostly non verbal. I closely observe what they like to play with and plan and provide resources as appropriate. They have recently began to display the container/enclosure play schema and I have planned activities that support this new pattern of behaviour, their favourite currently is putting plastic spoons in plastic cups. I look forward to working with older children when we can discuss ways that they think we can improve our play and what they would like to explore and learn. 

\chapter{Effectiveness of Leadership and Management}

\section{Managing the setting}

I pride myself on promoting an atmosphere that encourages children to thrive and excel. I create a warm home-from-home environment which allows children to explore, learn, discover and socialise. I always tailor activities to each child's individual needs and use planning effectively to enable children to flourish and achieve independent of their peers. Each child has unique next steps that are designed for them from careful observation to enable them to progress through the EYFS learning and development pathways. In 2016 I identified methods to improve my planning and implemented them, seeing an almost immediate improvement in my efficiency, improving sharing techniques with parents, encouraging their participation and creating a more accountable, circular approach to my planning. 

I have very high expectations for children, believing that it is of utmost importance to enthuse and motivate children's play and exploration both indoors and out. I take a personal responsibility in their development and work with them to enable them to achieve their full potential. I do this by talking to children to encourage a sense of wonder about their environment and to improve their language skills. I use parallel play and like to play by example in order to interest children in activities. For instance, I have previously planned activities in which I join a child playing with toy cars in order to model language and encourage them to use sounds during play. I also understand the importance of pulling back and allowing children to explore without intervention so that they can use resources in a unique way that an adult may not have thought of. They can also play freely without any pre conceived notions of a resource's intended use. I find “invitations to play” are great for this. An activity I have done in the past, for example, is home-made scented play dough, some fake leaves, some sticks and acorns (when age appropriate) Children may use them for the sensory experience of exploring the dough, they may create figures or objects for role-play, such as a birthday cake, they may use the small objects as counting props etc. Open ended resources are a fantastic way of allowing a child to use their imaginations and to lead their own learning in a direction that you may not expect. I like to find as many taste-safe sensory/messy play materials as I can. In my setting, children have played with green, dyed spaghetti, coloured rice, tapioca pearl balls, basil seed water beads and home-made husk slime. I'm constantly looking for more materials like this; they are very popular. I believe that it is important to know which teaching style each child responds to and to adapt your care appropriately; some children prefer activities outdoors, some prefer to have a calmer and quieter approach etc. In my opinion, I really get to know the children in my setting and strive to use a method that will enable them achieve their full potential. I am frequently striving to improve my methods of teaching and love to get inspiration from other child-carers and activity bloggers.

\section{Training and continuous improvement}

As discussed above, I am constantly seeking out new training opportunities and pride myself on my love of improving my knowledge and awareness of all things that could be beneficial to my setting. 
When appropriate training sessions are unavailable via my local authority, I find my own online training in a variety of areas and record them in my Continuous Professional Development (CPD) folder. In this folder, I keep a running note of all training that I undertake, when I did it, who provided the training and how long it took. I also keep self-assessment records in this folder, discussing what key issues were covered in each training session and how I can put my learning into practice. I also note any further reading that I need to do and how much approximate study time I have achieved. My CPD folder also holds a record of all professional discussions that I participate in. I think CPD is supremely important and am proud of the effort that I go to to remain aware of relevant information and new ideas. 

In my personal time, I enjoy to chat online with other Childminders, getting ideas for activities and looking for any way to ensure that my setting is exercising best practice. Other Childminders so far have been useful for a soundboard for new planning concepts, recommendations for resources for my setting, sharing new government policies and documentation, good activity ideas for specific play schemas, etc. I particularly like to visit the Childminding Help Forum  and the Independent Childminders group on Facebook. I have an online community of Childminders that I met at my “Preparation for Registration” training and I have found this very useful and a good source of advice, particularly when starting out after registration. There are also a lot of Childminders that meet up at the “story-rhyme time” session at the library and it is nice to exchange ideas and information with them. I have found that the children love to socialise and play together. In my personal time I have also read relevant documentation such as “The statutory framework for the EYFS”, “The early years inspection handbook”, “The common inspection framework”, “Getting it right first time”, “Inspecting safeguarding in early years, education and skills settings”, and others. I feel it is important to remain informed and to refresh your memory frequently. The Foundation Years website has also been a fantastic source of information and interesting reading. I have also made sure to remain in touch with my Local Authority's “Starting Life Well” (Early Years) team. They are always there to lend a hand and to offer advice. I feel this type of working in partnership is very important and a great way of remaining informed, getting advice and giving back to the Childminding community as well.

\section{Working in Partnership}

To successfully evaluate the quality of my provision I have discussed above how I like to work in partnership with parents and frequently ask for their feedback verbally and via questionnaires. If a parent has a request or a suggestion for improvement then I act on them immediately, as I find their feedback invaluable to improving my setting. I have had discussions with parents on a wide range of subjects, if I have been able to offer advice or guidance. I have offered advice on introducing independent feeding, vitamins, phasing out formula milk, sleep regression, behaviour management, good play activities and more. I have given resources (such as homemade play dough and slime) for them to take home to promote a continuation of learning in the child's home environment with their parents and extended family. I see it as my responsibility to have a positive effect on children at all times, not just during the time that they are in my setting. Working in partnership with parents and having an exemplary two way communication system is a vital way to ensure this and I make it very clear to parents that I am always here to help them and to offer advice. 

\section{Evaluation through reflecting on child development}

I also evaluate the quality of my provision by observing how children develop. A child's development is one of the best ways to ensure that your practice is successful and effective. If a child isn't progressing as expected along the pathway of the EYFS then this can be an indicator that your teaching methods may need to be re evaluated. Inside each child's Learning Journey, I keep a record of every observation that I make with respect to each aspect of the EYFS learning and development areas. In the past I have compared multiple children's observation matrices and looked for any area where observations have been less frequent than others. If I have been able to detect an area like this in the past then I have altered my planning to ensure that activities are appropriately encouraging learning in this particular area. As mentioned above, I also tailor each child's individual planning to encourage them to progress to the best of their ability. I use their observation matrix as a good tool to help do this alongside their individual learning and development trackers.

\section{My learning environments}

My indoor learning environment has been designed to be as accessible as possible for the age group for whom I provide early years care. I have previously lay out a variety of toys and resources on a mat for young children every day and have baskets at their height that are labelled with photographs and text for them to access freely when they want that particular resource. The musical instruments and ride-on toys are particularly popular! A successful resource that I have made is a treasure basket, full of different materials to explore with all of the senses', this can often entertain for a long while. My reading area has recently been redesigned to be even more accessible for the child in my care, with comfortable seating that is of an appropriate size and a new child sized sling bookshelf. So far the improvements have been very popular. I also provide a selection of board books on a very low shelf for free access for very young children. I have also recently made an age-appropriate mark making area in order to improve access to these materials. This has encouraged frequent use and enabled young children to be able to mark make without requiring intervention. I have also purchased some large dinosaur figures and log building blocks to improve imaginative play. All three of these improvements have been made as a result of the action plan that I created during my last update of my Self Evaluation Form.

My outdoor learning environment has been improved by an inventive use of storage boxes which enables children to access sand, water and messy play activities much more easily. The slide and garden swing are also very popular. Due to a child's love of splashing and water play I am currently reserving resources to make a water wall, which I believe will be very popular. This will also incorporate text into the outdoor environment, that is relevant to the children's play. I am also planning to improve sand play with better tools for digging and filling containers. I am greatly looking forward to implementing these changes. As discussed previously my planning is very child led and follows children's interests whilst encouraging them to progress in all aspects of learning and development. I like to use outings (such as the library, park and play centre) and at home activities to promote children's learning. The park and play centre are great places to encourage physical development and the library is a great area to explore - and to nurture a love of books from a very young age. Following children's interests is a great way to encourage learning as children are much more likely to participate in an activity if it involves something that particularly piques their interest. Good examples of this are incorporating a play schema that a child has demonstrated a fondness of, such as a containment/enclosure schema and using items during popular water play to discuss size and capacity.

All of these ideas for improving my setting will go on my action plan and will be tackled within a realistic time frame. I'm very excited to put these ideas into practice!

\section{Planning}

Planning done in the setting has different levels. Individual planning is very child led and comprises interesting ideas that are unique to that child (for instance, any activity that will allow them to learn via a preferred schema) and next steps from their observations. Weekly planning is essentially a weekly diary that is shared with parents and acts as a reminder for planned outings and activities from children's individual planning. Medium term planning is done when an event has a multitude of ways that it can be celebrated and works as a collection of good ideas to cover every area of learning and development, from these I pick and choose activities based on what is appropriate for the individual child. Long term planning is added to rather infrequently and is essentially a calendar of religious festivals and different events that occur annually. I have found this planning system to be very useful and easy to maintain, due to it being updated frequently, as a working document. This enables me to do my job to the best of my ability as I have an easy to locate reminder of what we need to achieve and what activities we are doing to promote learning that day. 

My weekly planning sheets ensure that the activities I intend to do will cover a wide range of EYFS learning and development areas and are also relevant to children's next steps and interests. These are stuck up on my parents noticeboard, encouraging parents to share their opinions on the activities.

\section{Starting Points}

To ensure that I have reliable starting points all parents are given a copy of “Early years outcomes” before the child starts at the setting and are asked to highlight the statements in there that they believe their child has demonstrated. I then spend the first month of a child's time here carefully observing them to gather evidence for these statements and note them on the child's development tracker alongside their parent's contributions. This ensures that I am able to monitor a child's progress from the beginning of their time with me and am able to ensure that all activities will be of benefit to them, encouraging them to move down their own pathway of learning, with their own next steps.

\section{Equal opportunities and British values}

I promote equal opportunities as it is important to recognise that every child is unique and an individual. My setting aims to value and celebrate diversity and recognise individual differences whether they be race, religion, gender, age, physical ability or dietary beliefs. I ensure that I treat all children with equal concern and respect to meet their individual needs and give them the opportunity to reach their full potential. I always help the children to feel good about themselves and others, by celebrating the differences which make us unique. I offer a wide range of resources (books, figures, decorations etc.) that reflect and celebrate diversity and celebrate and observe a variety of special dates from many religions such as Eid, Christmas, Chinese New Year, Diwali etc. This is done through many ways such as cooking, baking, music, crafting,artwork etc. I encourage the children in my care to learn more about their own culture and to find out about the culture and religions of other children from around the world, through fun activities. This leads to them having a healthy respect of each other’s differences and to value everyone as an individual. In the future, if a child has any additional needs or disabilities that require me to have further training this will be completed as soon as is realistically possible. I will work alongside parents/carers and outside agencies to ensure that each child’s needs are met at all times. I encourage parents/carers to share any festivals or special occasions which may enhance the children’s learning and understanding.

I actively promote the fundamental British Values of democracy, rule of law, individual liberty, mutual respect and tolerance for those with different faiths and beliefs, as they are already implicitly embedded in the EYFS. I understand that I have due regard to the need to prevent people from being drawn into terrorism and to protect them from radicalisation and extremism. I will do this by using all aspects of the EYFS. When utilising personal, social and emotional development children are encouraged to share, take turns and collaborate. Children are given opportunities to develop enquiring minds in an atmosphere where questions are valued. Children are encouraged to talk about their feelings, for example when they do or don't need help. They are encouraged to know that their views count and that their input is valued. When age-appropriate, I will demonstrate appropriate democracy in action, a good example of this will be to ask children to share their views on what toys and resources we should use, possibly with a show of hands.

\section{Policies and compliance}

I have a behaviour management policy that I follow and it includes some basic house rules that are enacted age appropriately. Primarily it is important that I myself am a good role model and don't expect children to behave in a way that I myself would not. It is important for children to learn right from wrong - and I assume some responsibility for this - as it is a vital part of their early years education. Like everything else, as a childminder it is very important to work in partnership with parents (and any other settings a child may attend) in this regard. I will work to ensure that children have a positive sense of themselves and provide opportunities for children to develop their self knowledge, self-esteem and increase their confidence in their own abilities. I have found great success in this by making photo-books of relatives for young children in the setting. It makes them feel at home and helps for them to develop a sense of self awareness. I will also encourage children to have confidence in their own abilities by allowing them to take safe risks (such as climbing at the park, supported for safety). I do not tolerate bullying and believe that early intervention is very important in preventing it. Children are encouraged to have kind hands, feet and words from an early age and I always encourage children to work well together, respecting one another and sharing. Bullying will always be tackled in an age appropriate way. 

I frequently read the Statutory framework for the EYFS in order for it to become second nature and to ensure that I comply with its requirements completely. As discussed above I also read government documents to ensure that I am aware of any new requirements or examples of good practice. I frequently check my practise to ensure that I am complying with the safeguarding and welfare requirements and the learning and development requirements. This means frequent risk assessments and constant vigilance. I have a robust and well thought-out safeguarding policy to refer to any time I need to. I am aware that everyone has a duty to protect children and I have taken a variety of courses on different aspects of safeguarding and child protection. I comply with government regulations for ICO registration and ensure that I update this annually. I also make sure that every three months I refresh my food hygiene knowledge and do a quarterly assessment.

My policies and procedures are available both on my website\footnote{http://www.karlieschildminding.co.uk} and in my setting. All parents read and agree to them before signing contracts with me. I find that it is immeasurably helpful to have written policies to communicate with parents before they start at the setting. This means that they are completely aware of our way of working before they choose to come to us and helps to prevent any disagreements that could arise later. I always stress the importance of working in partnership with parents and explain openly how much I encourage a two-way system of communication and information sharing. In the near future I am going to write three new policies, in order to improve this information sharing. These are a “Potty Training Policy”, a “Sleep Policy”, and a “Sun Protection Policy” This has been added to my action plan. *See updates

\section{School readiness \& next stage of learning}

I understand my role in helping children to transition to other settings and/or to be school-ready and am fully prepared to liaise with appropriate settings to work in the best interests of the child. I understand the importance of encouraging age-appropriate independence in preschool age children to prepare them for Reception class and in using observation, assessment and planning effectively so that they are on their way to achieving the early learning goals by the time they finish their Reception year. When children have left my setting in the past I have sent on their Learning Journeys, most recent end of term assessment, photos and my own contact details in case they need to contact me for any information regarding the child's care that will help to ease their transition into the new setting. I feel personally invested in each child's development so I will always ensure that I enable their new setting to take over observations, assessment and planning as easily as possible.

\section{Future plans}

My vision for the future is laid out in my action plan. I have plans to improve my indoor and outdoor provision, by purchasing new resources and redesigning old ones. I want to plan new and even more exciting ideas that enable the children in my care to make progress and achieve their full potential. I want more activities that excite, engage and educate. I plan to make new water play provision, will make a Dear Zoo story sack, get improved sand play tools and containers. I will discover more sensory/messy play resources that are taste safe, to ensure child safety. I look forward to having older children at the setting so that I can plan a wider range of activities enabling me to span the whole range of the EYFS. I intend to attend more training courses in the near future, and retrain for a local authority approved safeguarding course soon. I will also continue to seek out training opportunities and documents of interest to further my knowledge.

\section{Updates}

\begin{DESCRIPTION}
\item [June 2016] Wrote “Sun Protection Policy”, “Sleep Policy”, and “Potty Training Policy”
\item [June 2016] Read “Teaching and play in the early years – a balancing act?” Ofsted good practice survey to explore perceptions of teaching and play in the early years.
\item [June 2016] Read “Teaching young children to develop their communication skills: Nicola Phillips, childminder” Ofsted good practice document.
\item [June 2016] Read “Using the physical environment as a toll for teaching: Netherfield Primary School” Ofsted good practice document.
\item [June 2016] Attended Safe Sleep Guidance Training
\item [June 2016] Attended General Update Cluster Training
\item [August 2016] Arranged a meeting with the local Early Years team for them to assist with evaluating my setting. They said they were very happy with what I'm doing and signposted me to new safeguarding paperwork available through my Local Authority.
\item [September 2016] Read "Inspecting safeguarding in early years, education and skills settings" August 2016 edition.
\item [September 2016] Read and acted on guidance on how to update the setting's Safeguarding Policy in order to implement the changes suggested by the new edition of the "Inspecting safeguarding" document.
\item [September 2016] Researched on the drawbacks of using Early Years Outcomes (EYO) as a checklist. Read a few blogs and articles detailing how it is best practice to see EYO as a typical guide, rather than a definate list of milestones to hit.
\item [November 2016] Attended General Update Cluster Training
\item [November 2016] Updated Paediatric First Aid Training
\item [November 2016] Read "Unknown children- Destined for disadvantage"
\end{DESCRIPTION}

\chapter{Quality of Teaching, Learning \& Assessment}

Since becoming registered in 2015 - within the early years - I have worked primarily with children under two years of age. I have become extremely familiar with the appropriate EYFS learning and development age bands for this area. I firmly believe that each child should be regarded as an individual and that their development tracker should be appraised independently of any others. I don't allow myself to be limited by these age brackets, just guided by them. This means that I am perfectly happy to note when a child is achieving above what is expected, which I have found can certainly happen from time to time. I have very high expectations of what children can achieve and believe in challenging them whilst being completely supportive and encouraging.

\section{Assessments \& Planning}

At the end of each school term I assess children individually by noting on their trackers every statement from “Early years outcomes” that they have been observed achieving, to give an indicator of their current level of development. I transfer this information from their tracker to an end of term report sheet that I share with parents. This report has next steps for each of the seven areas of learning and development to inform parents and carers of what we are intending to cover next term and what they can enjoy working on at home. These next steps are added to the child's individual planning sheets and are a good jumping off point to plan for that child's learning from the beginning of the next term. Some next steps are completed quickly and others require a term or even more to achieve, either way we take the activities at the child's pace and try a variety of ways to engage the child in their learning. For example, toy animals, cars, dolls, play phones and trains are all different methods of encouraging noises whilst promoting babies to use sounds during play and are also good activities for an adult to model speech. I am very happy with how my subsequent planning system allows me to assist children to achieve their potential with maximum efficiency. As children's next steps are decided from their tracker and observations it makes my entire planning system very child led. This circle of observation-assessment-planning means that children are consistently monitored. This is very useful in identifying any children who need extra support. I am yet to have a child at the setting who requires extra support, but I am confident that my assessment system will result in me being able to respond quickly and efficiently to anything that arises.

\section{Working adaptively}

At the beginning of 2016 I identified a flaw in my paperwork, that it was taking a long time to get children's photographic evidence into their learning journeys. I purchased a small, bluetooth, photo printer that prints small, sticky back photos that can immediately be stuck on to observations and filed straight away without waiting for them to be developed or printed.

\section{Two-year Check}

I have not yet had to complete a two year check but have received appropriate training on this matter. As the child I have on my register has been here from a young age I will complete the check when they are early into their second year, to ensure that the documentation can be shared with their health visitor/community nursery nurse when the time is appropriate. I understand that best practice is to get the most up to date document that my local authority provides and to complete it with information from the child's tracker in conjunction with their parents. I understand that early intervention is of the utmost importance and am aware of my responsibility to share with parents any concerns I may spot whilst completing this document.

\section{Characteristics of effective learning}

I understand that it is not just what children learn that is important but also how children learn. The activities that I plan support children to develop the characteristics of effective learning, as does the environment that I have created at my setting. I make sure that my activities are varied, ranging from messy play to small world play to water play to open ended resource play etc. I encourage children to explore my setting and the places we visit and enable them to develop their own methods of learning. I always feel a sense of pride when I see children seeking challenges, showing curiosity or taking a risk. Whenever I observe a child demonstrating a new characteristic of effective learning for the first time, I note it on their observation. This is then immediately transferred to their tracker into the area designed to record these instances. If a child shows a particular tendency to one learning method, then I use it in order to encourage them to learn a new skill. For instance, if a child begins to be more confident exploring a familiar location, then I will plan for us to visit a new location with different opportunities for them to develop physically.
I have discussed previously how I pride myself on working in partnership with parents and carers. 
I see it as my personal responsibility to have a positive impact on children at all times, in and out of the provision. I have previously provided resources to take home (e.g. home-made taste safe slime and home-made play dough) to promote a continuation of learning in the child's home environment with their parents and extended family. On occasions such as World Book Day I have sourced books and acivity packs from the local library to send home with children to encourage and enable their parents to read, create and play with them. I have also shared information with parents about joining the library and given them a childrens sticker book designed to track how often they are going to the library and reward children for borrowing books with their parents.

\section{Working in partnership}

I like to help parents to understand how they can be involved in helping their child to learn. Working in partnership with parents and having an exemplary two way communication system is a vital way to ensure that I am supporting a child's learning both at my setting and at their own home. Parents have commented favourably on the “Today I have enjoyed” section of my daily diaries as it gives a good overview of what their child has enjoyed learning, exploring and playing with. My comments have affected what gifts a child has been given on special occasions and what games extended family are encouraged to play with them when they are their carers. I believe it is important to put all necessary information in the daily diary/communication book for young children to share this with their parents. Conversely, I also encourage and greatly enjoy when parents write back in the communication book and share what activities their child has been partaking in, what new things they have learned and any changes in their routine or behaviour. It is vitally important to me that parents are involved in the learning process and that information goes both ways as it greatly helps me to provide the best tailored care and teaching possible. 

\section{Equal opportunities}

I believe that it is important to promote equality of opportunities and diversity through teaching and I do this by introducing children to a wide variety of festivals, books, figures and environments. It is important to discuss openly and honestly with children how people come in many different shapes, sizes, colours, etc. and how we are all unique and special. Children need an expanded world view and I have previously discreetly discussed with curious children, individuals who are physically or visually impaired whilst we are out in the local area. I also find music from a variety of cultures is a great way to begin to introduce babies to the customs and traditions of others, as well as sampling food from other countries, where appropriate. The local library has a wonderful area on people with differing abilities and this is a great section to explore. In the future, I plan to purchase a large play figure set of people with differing abilities to incorporate into small world play. I also plan to increase my supply of books celebrating different cultural and religious festivals, so I can have them to hand and build on the ones that are available at the library. I look forward to having older children at the provision so we can more deeply explore customs and traditions of other cultures - including traditional dress, food, language, poetry, music, traditional dancing and flags. It is important to expose babies to these things from an early age, but it will have much more of an impact as they get a little older. 

\section{Holistic open-ended play}

I like to give children the freedom to play and explore and enjoy to design activities that are open ended, allowing for freedom of imagination and creativity (as discussed in my love of “invitations to play” previously). An offering of homemade slime and gold coins has previously turned into a counting and sorting activity. Open ended resources are almost limitless in ways that a child can use them and babies at my setting have loved to explore the evolving treasure basket here, as well as a great sensory activity, objects are turned into musical instruments and basic role play props amongst other things. I researched Montessori play techniques and read about the benefits of holistic, unlimited play. I designed the treasure basket around what I had learned and shared this information with parents. I have observed that large ride-on toys and push-along toys can have an impact on a child's ability to think critically, as they have to consider many options and approaches when navigating through different rooms and around furniture. These resources are always available at my setting - and are very popular. Open ended resources are also great way of discussing age-appropriate mathematics development. A sand or water tray is a wonderful place to discuss the differences between big and small objects and to introduce the concept of capacity. 
We sing a lot in the setting, as I feel that singing is a fantastic way of giving young children a well rounded variety of learning opportunities. Songs cover counting, letters, body parts, animals as well as being a great way to encourage babies to develop speech patterns and to be exposed to rhyming at an early age. Nursery rhyme repetition is a great way of encouraging babies to experiment with their first words. 

\section{Encouraging Prime areas of learning \& development in very young children to create a foundation for learning}

As I work predominantly with very young children I understand the importance of focusing on the prime areas of development in order to build a good foundation for learning that will then affect the specific areas of development. I ensure that all of my activities include a holistic approach, for instance any activity can be used as an opportunity to encourage personal, social and emotional development if it involves positively interacting with other children, sharing, taking turns etc. The aspects of making relationships and managing feelings and behaviour can be nurtured in almost any activity. Some of my most rewarding time with young children so far has been encouraging them to learn aspects of physical development, such as rolling from back to front and taking first steps. These are very exciting milestones and I am privileged to be able to share this task with parents. A lot of time has been spent building up young children's sense of security and confidence to enable them to take those first steps. We have also worked on using both hands whilst playing and other actions that perfect their moving and handling skills. More recently we have been focusing on communication and language skills. We have been singing a lot of songs and encouraging children to copy sounds and use sounds during play. We are also working on recognising familiar words and are making happy progress. This, in particular, is an area where I like to work in partnership with parents as I am very eager to hear how children are progressing at home. 

\section{Adapting resources \& spontaneous planning}

Occasionally I see a child playing with a resource and realise that the activity can be adapted to suit their play more appropriately and to develop their learning. A good example of this is when a child has previously been playing in the sand, pouring the sand from container to container but then prefers to try and build sand castles. The resource needed to be altered to enable them to do this, so I added water to make it possible. I then went on to model sand castle building and the children rapidly preferred destruction to construction, so I adapted again and made many sand castles for them to knock over.

\section{The importance of routine}

An adaptation of my daily routine had to be made when I first had babies attend the setting. Lunch had to be moved earlier to allow for the fact that they would be too sleepy to wait until the planned time. Our whole daily routine is now planned around this requirement. Routines and predictability are very important to create a secure environment where children feel comfortable to explore, discover and learn. In my experience a good routine can diminish unwanted  behaviour, as children feel more comfortable with stability. Good routines also allow young children to progress learning due to repetition. For instance a young child in my setting knows the area in which we sit together and they have their bottle of milk, meaning that as soon as they see the bottle they make their way over to the area. 

\section{Starting points \& a child's first few weeks here}

I have discussed previously my method for obtaining starting points from parents; how all parents are given a copy of “Early years outcomes” before the child starts and asked to highlight the statements in there that they believe their child has demonstrated. I then spend the first month of a child's time here carefully observing them to gather evidence for these statements and note them on the child's development tracker alongside their parent's contributions. This ensures that I am able to monitor a child's progress from the beginning of their time with me and am able to ensure that all activities will be of benefit to them, encouraging them to move down their own pathway of learning, with their own next steps. It is obviously not good practice to plan activities for a child until you have spoken to their parents and observed what they can or cannot do. This is the best way to make sure that the activities you plan will enable the child to progress to the best of their ability.

\section{Updates}

\begin{DESCRIPTION}
\item [December 2016] Sent home homemade multi-sensory resources to enable home learning over the school holidays.
\end{DESCRIPTION}

\chapter{Personal Development, Behaviour and Welfare}

\section{New children in my setting}

I ensure that when children begin at the setting we have quality one-to-one time in order for them to develop a secure emotional attachment with me, their key worker. I also facilitate a child's settling in by asking parents on their child's “All About Me” form if their child has any favourite songs, activities, comforters or for very young children how they are held to be comforted. In our provision we have two settling in sessions before a child starts here. The first session their parents are encouraged to stay, but can feel free to take a back seat to allow their child to explore freely. The second session the parents are encouraged to leave, as long as they feel comfortable doing so, so we can see how the child feels about being here without them. It is important to tailor care for each child so that they are supported during this time. So far all children who have started at the setting have settled quickly, and been perfectly happy here by the end of their first week.

\section{Managing behaviour}

In order to encourage children to learn how to behave well I have a robust behaviour management policy and house rules. These are simple and are always applied age appropriately. My behaviour management policy is available for viewing on my website (which I have linked to previously) I promote positive behaviour by using these house rules in combination with the managing feelings and behaviour aspect of personal, social and emotional development to support children to develop self control, cooperation and respect.

\section{Strong bonds and attachments}

Children are encouraged to play alongside their peers in order to develop good relationships. We emphasise the importance of kind hands and sharing. My daughter is very close with the child that I currently have at the setting, they truly are good friends who are always excited to see one another. This is a good example of how children are able to form appropriate bonds and emotional attachments here. I am also able to comfort the child easily if anything ever happens to unsettle them. They enjoy one to one time with myself when they have a bottle or when we sing songs and read stories together. 

\section{Safe risks \& independence}

Children are encouraged to take risks, but within safe boundaries. I am always very vocal with children from a young age about what dangers are to be avoided and of anything we need to be extra careful with. It is important to teach children how to learn to assess their own risks and keep themselves safe. The park is a great place to do this. Children learn to explore carefully and to make judgements on their own safety whilst navigating the play equipment. It is important for children to learn about their own personal safety, and managing risks is a big part of that. For example, I frequently discuss road safety when out and about with children. 

\section{Child involvement \& wellbeing}

I understand how necessary it is for children to thoroughly enjoy what they are doing. When a child is enjoying themselves it increases their level of involvement with the activity and facilitates an atmosphere where they can learn and grow. Children are born wanting to learn, curious about their surroundings and looking to people to support them, to teach them to be successful learners. Children can not be expected to progress in their development if they are not happy with what they are doing. It is important to look out for the well-being of children and in the future I plan to make a note on every observation of what I think their well-being and level of involvement is with that particular activity. (Based on “The Leuven Scale of Well-being”) This has been added to my action plan. *See Updates

\section{Healthy eating \& teaching children to keep themselves clean \& healthy}

I have a healthy eating policy that I give to parents before they start here and the menus that I have designed for the setting are very well thought out to ensure that there is a varied diet every day and that children are getting enough of all of the major food groups as well as a wide variety of fruit and vegetables. Everything is healthy and home-made. I only offer water and milk to drink (and extremely rarely fresh fruit juices) We do not have sweet treats (other than fruit and natural yoghurt) Friday is designated as a baking day and I have a wide variety of recipes that do not contain any refined sugar. I have done training on “Nutrition in the Early Years” and this is an area that is very important to me. 

When it is appropriate children are taught about how important it is to keep themselves healthy, and how food and exercise play an important role in this. Children are encouraged to try a wide variety of healthy foods and to explore vegetables that they may not have tried before. It is important to give children a variety of different tastes from a very young age in order for them to develop healthy habits that will last a lifetime. Play frequently involves a high level of activity and physical movement and we enjoy to play in the outdoor space and at the park, where children can practice running, climbing and finding new ways to move. Children are also encouraged to maintain a good level of personal hygiene and wash their hands frequently, as well as wiping their faces after eating, if necessary. Older children have their own towels whilst they are at the setting, and know which one is theirs. Very young children have their hands wiped before eating and are cleaned thoroughly afterwards. Nappy changes are always done hygienically on a mat that is cleaned in between uses with anti bacterial wipes. I wash my hands after each nappy change. After a nappy change children either have their hands wiped or are encouraged to wash their hands, based on their age and ability. Children understand where nappies are changed, and are used to this part of their routine. Children's health and self care development is promoted throughout our daily routine, such as using cutlery at meal times, learning to drink independently, or starting to learn how to put on their own coats when we are getting ready to play outside.

\section{Personal, social and emotional development \& encouraging independence}

I understand the importance of frequently planning activities that cover the personal, social and emotional area of learning and development. This ensures that children get a firm foundation in how to make friends, play well with peers and gain a sense of self awareness. I provide a comfortable environment where children feel free to explore and to develop their own confidence through making their own decisions, gaining independence and taking safe risks. Children are happy here and thoroughly enjoy what they are doing and this is evident through the progress that they are making in their learning. From a young age children are encouraged to gain independence, feeding themselves, holding their own cups, etc.  This leads to other self care such as toileting and dressing. Children are encouraged to choose their own toys and activities and to be responsible for helping to tidy up resources when they are done with them. Encouraging independence and self care in this way means children are more prepared for the transition into their next stage of learning, including school when appropriate. School readiness is very important and I see it as my responsibility to promote progression throughout the areas of learning and development so that all children are on the pathway towards achieving the early learning goals. Children are taken to places where they are encouraged to explore and discover new things, though I always make sure to remain nearby as a safe base for them to return to if they require reassurance. Independent play is how children learn to use their imagination, I have discussed previously my love of open ended resources, that allow children to direct the activity in a way of their own choosing. I frequently take children to meetings and groups where they are encouraged to socialise in larger numbers, this encourages their confidence in social situations and cultivates an environment where children are able to converse and play comfortably without requiring adult intervention.

\section{Ensuring children's frequent attendance by working with parents}

All children that have been at the setting since I registered in 2015 have attended regularly. This is another area where I think it is highly important to work in partnership with parents. Whenever children have a reason that they cannot attend, parents have always been very communicative and informed me before hand. It is important to build up this kind of relationship with parents, where both sides value the sharing of information. I keep a very accurate register at the setting that monitors what time the children enter and leave. Parents sign this register at the end of the week, agreeing to its contents.

\section{Teaching social responsibility in modern Britain}

A wide variety of activities and types of play are used to facilitate learning. We have a toy kitchen as well as a role-play basket full of props that children can use during their imaginative play, such as a teapot that teaches manners and a picnic basket that encourages sharing and teaches shapes. We celebrate events and festivals from the United Kingdom and from around the world, such as St Patrick's day, St David's day, Chinese New Year, Christmas, Diwali etc. This helps to teach children about the world in which they live and prepares them for life in modern Britain. I believe it is important to introduce children to a variety of cultures, religions and spiritual paths to create a level of understanding and empathy for others as well as expanding their world view. Small world toys are a great way for children to play out scenarios that teach them about moral responsibility and about the importance of kindness, for example, dolls and figures who are sharing and resolving conflicts peacefully.

\section{Updates}

\begin{DESCRIPTION}
\item [June 2016] Started to make a note on every observation of what I think children's well-being is with that particular activity, (Based on “The Leuven Scale of Well-being”) in order to remain vigilent to children's happiness and well-being.
\end{DESCRIPTION}

\chapter{Outcomes for Children}

\section{Starting points \& a child's first few weeks}

I ensure that children's starting points are obtained before they begin at the setting. I do this by giving parents a copy of “Early years outcomes” (EYO) when we are completing paperwork. I sit down with the parents and ask them to highlight every statement that they have witnessed their child doing and make sure I am on hand to help with anything that they may not understand. This means that I am able to observe, assess and plan for children as soon as they start at the setting. When a child has come to me and they already attend another provision I have contacted the other setting to get a copy of their development tracker to guide their starting points with me. For the child's first month at the provision I watch them very carefully and gather as many observations as I can in order to get to know them and to get an idea of where they are at developmentally. After this month I assess them on their tracker and combine my observations with the starting points provided by their parents. These are colour coded, so it can be seen at a glance which contributions are from their parents and which are from my observations.

\section{Observations \& assessments}

I do observations frequently whenever I notice a child developing a new skill or demonstrating a new characteristic of effective learning. These are stored in the child's Learning Journey with evidence, if appropriate. At the end of each term I assess the children and complete their trackers, copying up the EYO statements that have been noted on their observations. I then look at the trackers, taking into consideration all 7 areas of learning and development, but focusing on the prime areas for younger children. From this I get a good idea of the progress they have made every term. I then copy this assessment up onto an end of term report and share them with parents. These reports have been very well received. Parents have found them incredibly useful and informative, the information has helped to influence activities at home, gifts for special occasions and has been used to communicate with health visitors at a child's one year check. I have previously discussed my planning system and how it is directly impacted by my observations of individual children and their next steps, as well as influenced by current events.

\section{Progress of children in my setting}

All of the children that have attended my provision have made good progress from their starting points, I enjoy watching children move from strength to strength. I find that with a wide variety of activities, a child led planning system, being observant and reactive to children's needs that learning through play is a natural result. I see it as my personal responsibility to raise outcomes for every child that attends my setting and to ensure that they continue to be completely supported and successfully challenged every term. I seek to improve myself and my provision through regular CPD, training and self evaluation. I reflect on my practice frequently to ensure that I am working in a way that will allow children to make the best progress possible. 

Since registering I have not yet had a child that has been assessed as falling behind 'typical behaviours' in EYO, but I am sure that with my system of assessments it will become clear very quickly enabling me to intervene and offer extra support to assist children to catch up, so that they are soon learning to the best of their ability. Completing children's trackers in the way that I do means that it's very easy to see if a child is falling behind in a particular area, allowing me to alter planning to ensure that this is addressed as quickly as possible.

\section{Children's individuality}

As I have discussed previously, I see it as imperative to regard each child's progress individually. All children should be encouraged to progress to the best of their ability and should be planned for based on their personal assessments. This means if a child is working at a level above what is typical of their age then their planning should reflect that. I have very high expectations of children and believe that they should be encouraged to achieve their full potential. If a child's planning is individual then it shouldn't matter that they are working at a higher level than expected. I ensure that all children are challenged and supported, whatever stage of learning they are at.

If children are encouraged to progress down the learning and development pathway in the EYFS then they should naturally be working towards obtaining the early learning goals at the end of their Reception year. This along with encouraging independence, as laid out in the health and self care aspect of physical development, should be giving them a strong start towards school readiness.

\chapter{Overall Effectiveness}

\section{Children's progess}

It is my experience that all children that attend my setting make good progress in their learning and development from their starting points. I have gone into detail previously about my child led circular observation, assessment and planning system, and how it is responsive to the needs and development of each child individually. As the system is tailored in this way I feel that it would be very effective at efficiently identifying if a child had a possible delay in an area of learning, allowing me to tackle this as quickly as possible and work in partnership with parents and other professionals (where appropriate) to find a solution that is best for the child. My setting works in a way that ensures all children are able to reach their full potential. 

I have discussed how I facilitate children's development to promote school readiness, by encouraging independence and by planning for children's learning in way that promotes them to move down the pathway of EYFS learning and development, working towards achieving the early learning goals at the end of their Reception year. 

\section{Children's wellbeing \& happiness}

I work very hard to make certain that children are happy here and always ensure that they are secure and safe. I have mentioned before my methods of settling in and my determination to work in partnership with parents to see to it that children settle here as quickly as possible so that we can create a nurturing, inspirational environment for them to thrive. I have introduced a new addition to my observations to make a note of children's level of well-being during activities in order to carefully monitor this area.

\section{Safeguarding children}

I frequently read the “Statutory framework for the EYFS” in order to ensure that I comply with the safeguarding and welfare requirements and review all of my policies annually, including the ones relating to child protection and safeguarding. I have done a wide variety of training courses relating to safeguarding and am always on the look out for more. I am a member of a Facebook group dedicated to sharing news articles related to the safeguarding of children in order to stay informed of any new risks or actions by the authorities. I understand that it is the role of everybody to protect children.

\section{Evaluation and improvement}

I am very dedicated to constant self evaluation and continuous improvement. I am always on the look out for ways in which I can better my provision. I seek feedback from parents and will also seek it from children when age appropriate. I judge my provision on the development children make and will always remain aware of any way in which I can greater facilitate an environment where each child can achieve their full potential and develop a good foundation of learning to enable them to thrive in their next stage of learning. 

I look forward to moving on in the future to achieve the points laid out in my action plan by further improving the setting with more detailed observations, improved resources and more free access areas for children.

\end{document}
