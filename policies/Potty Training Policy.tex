\documentclass[10pt]{article}

\usepackage{fullpage}
\usepackage{tabularx}
\usepackage{colortbl}
\usepackage{xcolor}
\usepackage{fontspec}
\usepackage{amsmath}
\usepackage{titlesec}
\defaultfontfeatures{Ligatures=TeX}
% Set sans serif font to Calibri
\setsansfont{Carlito}
% Set serifed font to Cambria
\setmainfont{Caladea}
% Define light and dark Microsoft blue colours
\definecolor{MSBlue}{rgb}{.204,.353,.541}
\definecolor{MSLightBlue}{rgb}{.31,.506,.741}
% Define a new fontfamily for the subsubsection font
% Don't use \fontspec directly to change the font
\newfontfamily\subsubsectionfont[Color=MSLightBlue]{Times New Roman}
% Set formats for each heading level
\titleformat*{\section}{\huge\bfseries\sffamily\color{MSBlue}}
\titleformat*{\subsection}{\Large\bfseries\sffamily\color{MSLightBlue}}
\titleformat*{\subsubsection}{\itshape\subsubsectionfont}

\usepackage{titlesec}
\titlespacing*{\subsection}{0pt}{1ex plus 1ex minus .2ex}{0ex}

\setcounter{secnumdepth}{0}

\setlength{\parindent}{0pt}
\setlength{\parskip}{\baselineskip}%

\pagestyle{empty}

\begin{document}


\section{Potty Training Policy}

I aim to work in partnership with parents when you wish to start potty training your child. It is important to note that every child is different and can not be hurried into potty training before they are ready. Many children will start to show strong signs that they are ready for potty training before they are 3 years old. A child's bladder capacity increases significantly between the ages of 2 and 3, so that by the time that they are 3 most children are able exercise bladder control and stay dry for longer periods of time. For some children this is younger and some can take longer than others, every child is an individual.

When you believe that your child is ready to be potty trained I will require you to start the process at home, during the school holidays if possible and if not then at the weekend. We can introduce your child to the idea gently, by putting them on a potty at regular times during the day and when they have their nappy changed. I will require your child to wear either 'pull ups' (which I can provide) or training pants until they are able to ask to use the toilet/potty before they require it and can also control their bladder/bowels a few moments beyond that announcement. Normal underwear can only be worn here once your child has demonstrated full bladder/bowel control and an ability to ask to be taken to the toilet/potty. If you decide to provide training pants then children will need to wear 'pull ups' for outings and nap times until they are fully trained. Whilst potty training children will need to wear suitable clothing that is easy for them to pull up and down, e.g. no dungarees or vests with poppers. 

Whilst your child is potty training I recommend that you supply at least 2 extra full changes of clothing, including socks, where appropriate. Soiled clothes will be returned in a plastic bag at the end of the day, complying with the requirements of Public Health England.

Here are some signs that your child may be ready for potty training:
\begin{itemize}
\item ``I can do it'' becomes a regular refrain -- showing that your toddler wants to become more independent.
\item They have regular formed bowel movements and may go red in the face and gain a very concentrated expression when about to go.
\item They have the dexterity to pull their own trousers up and down.
\item They are interested when you go to the toilet yourself.
\item They are dry for longer periods of time, up to 3-4 hours, showing that their bladder capacity and control are improving.
\item They are able to understand simple instructions.
\item They have started to recognise the sensations that mean they need to use the toilet and have demonstrated this either verbally or by holding themselves and grunting.
\item They may complain about a soiled nappy.
\item They may have started to remove their own nappy when they have wet it
\end{itemize}

Any accidents will be dealt with calmly and there will be no blame assigned or discipline given. Your child will not be made to feel as if they have done something wrong. Praise will be used excessively when your child uses the potty or toilet, and a lot of encouragement will be given. 

It is not advisable to start potty training during any major changes in your child's life, for example a new sibling or a move of house. If after a couple of weeks it is evident that your child is not ready best practice is to give them a break from it and try again in another couple of months. Please note the above are guidelines and we will discuss the individual requirements for your child when the time arrives. If you have any concerns regarding this policy then please discuss them with me.

Review Date: 31/08/2017

\end{document}

