\documentclass[10pt]{article}

\usepackage{fullpage}
\usepackage{tabularx}
\usepackage{colortbl}
\usepackage{xcolor}
\usepackage{fontspec}
\usepackage{amsmath}
\usepackage{titlesec}
\defaultfontfeatures{Ligatures=TeX}
% Set sans serif font to Calibri
\setsansfont{Carlito}
% Set serifed font to Cambria
\setmainfont{Caladea}
% Define light and dark Microsoft blue colours
\definecolor{MSBlue}{rgb}{.204,.353,.541}
\definecolor{MSLightBlue}{rgb}{.31,.506,.741}
% Define a new fontfamily for the subsubsection font
% Don't use \fontspec directly to change the font
\newfontfamily\subsubsectionfont[Color=MSLightBlue]{Times New Roman}
% Set formats for each heading level
\titleformat*{\section}{\huge\bfseries\sffamily\color{MSBlue}}
\titleformat*{\subsection}{\Large\bfseries\sffamily\color{MSLightBlue}}
\titleformat*{\subsubsection}{\itshape\subsubsectionfont}

\usepackage{titlesec}
\titlespacing*{\subsection}{0pt}{1ex plus 1ex minus .2ex}{0ex}

\setcounter{secnumdepth}{0}

\setlength{\parindent}{0pt}
\setlength{\parskip}{\baselineskip}%

\pagestyle{empty}

\begin{document}


\section{Behaviour Management Policy}

All children have the right to be cared for in a happy environment;
therefore it is important to ensure that all children know what is
expected of them and what the boundaries are. In order to achieve this
there are some house rules which set reasonable and appropriate limits
to help manage the behaviour of the children. (See below)

I help the children understand my house rules, which are realistic and I
am consistent in the enforcing of them. To help them be followed,
children of a suitable age will be asked to help write them. Children
will be reminded of the rules when necessary.

I will ensure that:

\begin{itemize}
\item
  What I expect from the children is reasonable and achievable,
  depending on their age and ability. Children's behaviour will always
  be dealt with age appropriately; ~
\item
  I make myself clear when giving an explanation of what behaviour was
  unacceptable and why;~
\item
  I am a good role model;~
\item
  I listen to what the children have to say; ~
\item
  I reward good behaviour;~
\item
  Humiliating treatment, physical punishment -- or threat of physical
  punishment -- are never used and~
\item
  Physical restraint is not used, unless it is necessary to prevent
  harm to themselves; other people or property. This incident will be
  recorded appropriately and parents will be informed. ~
\end{itemize}

There are several ways to deal with a child who is displaying
unacceptable behaviour. Depending on the age/stage of ability of the
child I will use different methods, these are:

\begin{itemize}
\item
  Distraction: Remove the child from the situation and give them an
  alternate activity. ~
\item
  Ignore: Depending on the situation I may ignore the unacceptable
  behaviour as I feel it is being done to get a reaction. It is much
  more effective to respond to positive behaviour than to negative
  behaviour. ~
\item
  Discuss with the Child: If the child is able to understand I will
  discuss their behaviour with them and try to get them to appreciate
  the consequences of their actions on others. I will inform them that
  it is their behaviour that is unacceptable, not them.~
\item
  The word ``naughty'' will never be used.~
\end{itemize}

These methods of behaviour management will be agreed with the
parent/carer before the child first attends the setting in order ~to
manage behaviour effectively and I will continue to work in partnership
with parents to create consistency for the child between their time at
home and their time at the setting.

Significant incidents will be recorded on an incident form, where it
will be discussed with the parent/carer that day. Both parties will be
required to sign the form once it has been discussed.

I am aware that it is within my role to guide children and deal with
small, everyday incidents as and when they occur, but I will always
inform you if your child has been involved in a significant incident.

\subsection{House rules}

\begin{enumerate}
\item
  We're kind to each other, using good manners, kind hands, feet and
  words. ~
\item
  We respect each other, are polite and listen to others talking.~
\item
  We look after our toys, equipment, belongings and furniture.~
\item
  We share our toys and activities and help to tidy away when finished
  with them.~
\end{enumerate}

Review Date: 31/08/2017

\end{document}

