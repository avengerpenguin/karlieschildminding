\documentclass[10pt]{article}

\usepackage{fullpage}
\usepackage{tabularx}
\usepackage{colortbl}
\usepackage{xcolor}
\usepackage{fontspec}
\usepackage{xcolor}
\usepackage{titlesec}
\defaultfontfeatures{Ligatures=TeX}
% Set sans serif font to Calibri
\setsansfont{Carlito}
% Set serifed font to Cambria
\setmainfont{Caladea}
% Define light and dark Microsoft blue colours
\definecolor{MSBlue}{rgb}{.204,.353,.541}
\definecolor{MSLightBlue}{rgb}{.31,.506,.741}
% Define a new fontfamily for the subsubsection font
% Don't use \fontspec directly to change the font
\newfontfamily\subsubsectionfont[Color=MSLightBlue]{Times New Roman}
% Set formats for each heading level
\titleformat*{\section}{\huge\bfseries\sffamily\color{MSBlue}}
\titleformat*{\subsection}{\Large\bfseries\sffamily\color{MSLightBlue}}
\titleformat*{\subsubsection}{\itshape\subsubsectionfont}

\setcounter{secnumdepth}{0}

\setlength{\parindent}{0pt}
\setlength{\parskip}{\baselineskip}%

\pagestyle{empty}

\begin{document}

\section{Lost Child Policy}

The care of your child is paramount and I always have risk assessments
in place to ensure that they remain with me and are safe.

However on rare occasions children can become `lost' in busy places and
therefore as a responsible childminder I have written a procedure that
will be followed in the unlikely event of this happening.

\begin{itemize}
\item
  I will immediately raise the alarm to all around me that I have lost
  a child and enlist the help of everyone to look for them.
\item
  If it is a secure area, such as a shopping centre, I will quickly
  alert the security staff so that they can seal off any exits and
  monitor the situation on CCTV.
\item
  I will provide everyone involved in the search with a description of
  the child.
\item
  I will reassure any other children with me, as they may be
  distressed.
\item
  After 10 minutes I will alert the police and provide them with a
  full description.
\item
  I will then alert the parents/carers to the situation.
\item
  I will inform Ofsted of this incident.
\end{itemize}

I take precautions to avoid situations like this happening by
implementing the following measures:

\begin{itemize}
\item
  Ensuring the children hold my hand or the pushchair whilst we are
  out.
\item
  Avoid going to places that are overcrowded.
\item
  On outings the children wear wristbands with my mobile number on
  them.
\item
  When the children are old enough I teach the children about the
  dangers of wandering off and talking to strangers.
\end{itemize}

Date:07/06/2014

~
\end{document}
