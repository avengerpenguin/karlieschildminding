\documentclass[10pt]{article}

\usepackage{fullpage}
\usepackage{tabularx}
\usepackage{colortbl}
\usepackage{xcolor}
\usepackage{fontspec}
\usepackage{amsmath}
\usepackage{titlesec}
\defaultfontfeatures{Ligatures=TeX}
% Set sans serif font to Calibri
\setsansfont{Carlito}
% Set serifed font to Cambria
\setmainfont{Caladea}
% Define light and dark Microsoft blue colours
\definecolor{MSBlue}{rgb}{.204,.353,.541}
\definecolor{MSLightBlue}{rgb}{.31,.506,.741}
% Define a new fontfamily for the subsubsection font
% Don't use \fontspec directly to change the font
\newfontfamily\subsubsectionfont[Color=MSLightBlue]{Times New Roman}
% Set formats for each heading level
\titleformat*{\section}{\huge\bfseries\sffamily\color{MSBlue}}
\titleformat*{\subsection}{\Large\bfseries\sffamily\color{MSLightBlue}}
\titleformat*{\subsubsection}{\itshape\subsubsectionfont}

\usepackage{titlesec}
\titlespacing*{\subsection}{0pt}{1ex plus 1ex minus .2ex}{0ex}

\setcounter{secnumdepth}{0}

\setlength{\parindent}{0pt}
\setlength{\parskip}{\baselineskip}%

\pagestyle{empty}

\begin{document}


\section{Sleep Policy}

At my setting I regard sleep as an important part of a child's daily care and well-being, from babies through to preschool aged children. Good sleep is vital for healthy brain development and it is in a child's best interests for them to be rested.

You will be consulted about your child's sleep routine when you complete the 'All About Me' document on your first visit. All parents will be advised that guidance suggests young children under 6 months old should be placed flat on their back to sleep. This reduces the risk of SUDI (Sudden Unexpected Death in Infancy) and is better for children's posture. I will outline all of the sleep arrangements available and will agree a plan with you, the parent. You will be asked if you have a preference to the amount of time your child sleeps, and the times during the day to ensure consistency in their routine as much as possible.

I will ensure that:
\begin{itemize}
\item No child shall be deprived of sleep at any time if it is necessary for them to have a rest. Parents' wishes will be respected but a child will not be woken up if they naturally fall asleep. I will leave a sleeping child for a minimum of 45 minutes in accordance with the human rights act 1998. Meals will be kept for a sleeping child and offered to them when they wake. 
\item Sleeping children will be placed in a clean, quiet and comfortable area on a flat mattress or travel cot.
\item Travel cots are provided for younger children and no cot bumpers will be used, nor will any pillows, quilts or unnecessary bedding. 
\item Older children that need a rest will be lay down on sleep mats and monitored regularly.
\item All babies and children will be monitored regularly whilst sleeping, to check that they are safe and well. 
\item Babies will be laid on their backs, with their feet to the foot of the cot, with bedding tucked in tightly up to chest height to prevent them from wriggling down under it.
\item Shoes, bibs, loose clothing, hair clips and anything that could be uncomfortable or pose a choking hazard will be removed before any child is laid down to sleep. Children will not be allowed to sleep with a bottle in their mouths.
\item Babies sleeping in cots will have their own bedding, not used by other children. This will be washed at the end of the week, unless prior to that it is wet or dirty. Older children will have their own blankets that will only be used by them. These will also be washed at the end of the week. Blankets are light, cotton, cellular blankets to prevent over heating and to allow air to flow through.
\item Children will be lay down to sleep in an area where they can be seen and/or heard, including through a baby monitor where appropriate.
\item I do not encourage babies to sleep in car seats or pushchairs, due to the recommended 'Safe Sleep' guidance and the  risks these pose to young children. If a child is napping in a pushchair or a car seat due to us being on an outing then they will be closely monitored.
\item If a child arrives at the setting asleep in a pushchair or car seat they will be removed, along with any outdoor clothing and placed on a flat bed, as described above. 
\item Parents will be informed if their child has a sleep at the setting.
\item Babies and children are welcome to bring their own comforters from home if this helps them to settle, as long as they are in line with 'Safe Sleep' guidance.
\end{itemize}

I will work in partnership with you and discuss how your child falls asleep at home, whether they need to be rocked to sleep or are best to be given space, whether they require a dummy etc. 
Safe Sleep Guidance:

\begin{itemize}
\item Room temperature for sleeping babies is between 16-20 degrees
\item No hats or outdoor clothing when sleeping
\item Babies aged 0-6 months to be lay down to sleep on their backs with their feet to the foot of the cot.
\item Blankets to be tucked in and only up to chest height
\item No being put down to sleep in car seats or pushchairs
\item Children should not be subjected to second hand smoke. It is best to fully change clothes after smoking, before holding a baby and to be aware of how long smoke can stay on the breath. 
\item If a child usually sleeps with a dummy then this should be encouraged in any other settings where they sleep. 
\end{itemize}

For parents' information, this is the recommended time for sleep for different ages:

\begin{table}[h]
  \begin{tabular}{|l|l|}
    \hline
    Age & Average sleep needs \\
    \hline
    1 to 4 weeks     &   15-16 hours \\
    \hline       
    1 to 4 months &   14-15 hours \\
    \hline
    1 to 3 years &   12-14 hours \\
    \hline
    3 to 6 years &   10-12 hours \\
    \hline
    7 to 12 years &   9.5-10.5 hours \\
    \hline
    12 to 18 years &   8.5-9.5 hours \\
    \hline
  \end{tabular}
\end{table}

Review Date: 31/08/2016

\end{document}

