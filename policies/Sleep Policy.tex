\documentclass[10pt]{article}

\usepackage{fullpage}
\usepackage{tabularx}
\usepackage{colortbl}
\usepackage{xcolor}
\usepackage{fontspec}
\usepackage{xcolor}
\usepackage{titlesec}
\defaultfontfeatures{Ligatures=TeX}
% Set sans serif font to Calibri
\setsansfont{Carlito}
% Set serifed font to Cambria
\setmainfont{Caladea}
% Define light and dark Microsoft blue colours
\definecolor{MSBlue}{rgb}{.204,.353,.541}
\definecolor{MSLightBlue}{rgb}{.31,.506,.741}
% Define a new fontfamily for the subsubsection font
% Don't use \fontspec directly to change the font
\newfontfamily\subsubsectionfont[Color=MSLightBlue]{Times New Roman}
% Set formats for each heading level
\titleformat*{\section}{\huge\bfseries\sffamily\color{MSBlue}}
\titleformat*{\subsection}{\Large\bfseries\sffamily\color{MSLightBlue}}
\titleformat*{\subsubsection}{\itshape\subsubsectionfont}

\setcounter{secnumdepth}{0}

\setlength{\parindent}{0pt}
\setlength{\parskip}{\baselineskip}%

\pagestyle{empty}

\begin{document}


\section{Sleep Policy}

At my setting I regard sleep as an important part of a child's daily care and well-being, from babies through to preschool aged children. Good sleep is vital for healthy brain development and it is in a child's best interests for them to be rested.

You will be consulted about your child's sleep routine when you complete the 'All About Me' document on your first visit. All parents will be advised that guidance suggests young children under 6 months old should be placed flat on their back to sleep. This reduces the risk of SUDI (Sudden Unexpected Death in Infancy) and is better for children's posture. I will outline all of the sleep arrangements available and will agree a plan with you, the parent. You will be asked if you have a preference to the amount of time your child sleeps, and the times during the day to ensure consistency in their routine as much as possible.

I will ensure that:
\begin{itemize}
\item No child shall be deprived of sleep at any time if it is necessary for them to have a rest. Parents' wishes will be respected but a child will not be woken up if they naturally fall asleep. I will leave a sleeping child for a minimum of 45 minutes in accordance with the human rights act 1998. Meals will be kept for a sleeping child and offered to them when they wake. 
\item Sleeping children will be placed in a clean, quiet and comfortable area on a flat mattress or travel cot.
\item Travel cots are provided for younger children and no cot bumpers will be used, nor will any pillows, quilts or unnecessary bedding. 
\item Older children that need a rest will be lay down on sleep mats and monitored regularly.
\item All babies and children will be monitored regularly whilst sleeping, to check that they are safe and well. 
\item Babies will be laid on their backs, with their feet to the foot of the cot, with bedding tucked in tightly up to chest height to prevent them from wriggling down under it.
\item Shoes, bibs, loose clothing, hair clips and anything that could be uncomfortable or pose a choking hazard will be removed before any child is laid down to sleep. Children will not be allowed to sleep with a bottle in their mouths.
\item Babies sleeping in cots will have their own bedding, not used by other children. This will be washed at the end of the week, unless prior to that it is wet or dirty. Older children will have their own blankets that will only be used by them. These will also be washed at the end of the week. Blankets are light, cotton, cellular blankets to prevent over heating and to allow air to flow through.
\item Children will be lay down to sleep in an area where they can be seen and/or heard, including through a baby monitor where appropriate.
\item I do not encourage babies to sleep in car seats or pushchairs, due to the recommended 'Safe Sleep' guidance and the  risks these pose to young children. If a child is napping in a pushchair or a car seat due to us being on an outing then they will be closely monitored.
\item If a child arrives at the setting asleep in a pushchair or car seat they will be removed, along with any outdoor clothing and placed on a flat bed, as described above. 
\item Parents will be informed if their child has a sleep at the setting.
\item Babies and children are welcome to bring their own comforters from home if this helps them to settle, as long as they are in line with 'Safe Sleep' guidance.
\end{itemize}

I will work in partnership with you and discuss how your child falls asleep at home, whether they need to be rocked to sleep or are best to be given space, whether they require a dummy etc. 
Safe Sleep Guidance:

\begin{itemize}
\item Room temperature for sleeping babies is between 16-20 degrees
\item No hats or outdoor clothing when sleeping
\item Babies aged 0-6 months to be lay down to sleep on their backs with their feet to the foot of the cot.
\item Blankets to be tucked in and only up to chest height
\item No being put down to sleep in car seats or pushchairs
\item Children should not be subjected to second hand smoke. It is best to fully change clothes after smoking, before holding a baby and to be aware of how long smoke can stay on the breath. 
\item If a child usually sleeps with a dummy then this should be encouraged in any other settings where they sleep. 
\end{itemize}

For parents' information, this is the recommended time for sleep for different ages:

\begin{table}[h]
  \begin{tabular}{|l|l|}
    \hline
    Age & Average sleep needs \\
    \hline
    1 to 4 weeks     &   15-16 hours \\
    \hline       
    1 to 4 months &   14-15 hours \\
    \hline
    1 to 3 years &   12-14 hours \\
    \hline
    3 to 6 years &   10-12 hours \\
    \hline
    7 to 12 years &   9.5-10.5 hours \\
    \hline
    12 to 18 years &   8.5-9.5 hours \\
    \hline
  \end{tabular}
\end{table}

Review Date: 31/08/2016

\appendix
\section{Frequently Asked Questions}

\subsection{What does Sudden Unexpected Death in Infancy (SUDI) mean?}
SUDI is a term used to describe the sudden and unexpected death of a baby. SUDI may be the result 
of a serious illness or a problem that baby may have been born with, but most SUDI deaths occur as 
a result of either SIDS (sudden infant death syndrome) or a fatal sleep accident.
The only way to find out why a baby has died suddenly and unexpectedly is to perform an
autopsy, review the clinical history and to thoroughly investigate the circumstances of death, 
including the death scene. When no cause can be found for the death it
is called SIDS.

\subsection{How can I reduce the risk of Sudden Unexpected Death in Infancy?}

Create a safe sleeping environment for babies and young children:

\begin{itemize}
\item Put baby on the back to sleep from birth
\item Sleep baby with head and face uncovered 
\item Avoid exposing babies to cigarette smoke before birth and after
\item Sleep baby in a safe cot and in a safe environment
\item Sleep baby in its own cot or Moses basket in the same room as the parents for the first 6-12 months
\item Breastfeed baby
\end{itemize}

\subsection{How much clothing/bedding does baby need?}
Babies control their temperature through the face. Sleeping baby on the back and ensuring
that the face and head remains uncovered during sleep is the best way to protect baby
from overheating and suffocation. Sleeping baby in a sleeping bag will prevent 
bed clothes covering the baby’s face.

If blankets are being used instead of a sleeping bag, it is best to use layers of lightweight blankets 
that can be added or removed easily according to the room temperature and which can be tucked 
underneath the mattress.

When dressing a baby you need to consider where you live, whether you have home heating
or cooling and whether it is summer or winter. A useful guide is to dress baby as you would
dress yourself – to be comfortably warm, not hot or cold. Ensure that baby is dressed appropriately 
for the room temperature. A good way to check baby’s temperature is to feel baby’s chest, which 
should feel warm (don’t worry if baby’s hands and feet feel cool, this is normal). Another way to 
prevent overheating is to remove hats or bonnets from baby as soon as you come indoors or enter a  
warm car, bus or train, even if it means waking the baby.

\subsection{Is it safe to sleep baby on a baby bean bag?}
No. A bean bag, defined as a material sack encasing a large quantity of polystyrene foam beads that 
is usually a pyramid-shaped sack used for seating, poses a suffocation risk to babies and small 
children if they inhale the beads.  Concern has been raised about the potential of some bean bags
being capable of contouring around a baby´s face, resulting in a risk
of suffocation.
 
\subsection{What is the safest way to sleep twins?}
The safest way to sleep twins is to place them in their own cot following the steps to safe sleeping. 
However, sometimes you may need to sleep twins in the same cot, for example when you
are travelling or visiting and there is insufficient room for two cots. If this is the case, place each
twin at opposite ends of the cot as this will minimise the risk of one twin covering the face of the 
other. When the babies are able to move freely around the cot, put
them to sleep in separate cots.

\subsection{Does dummy use reduce the risk of sudden unexpected death in infancy?}
Some studies have shown that using a dummy at the start of every sleep may reduce the association 
with SIDS and that stopping or inconsistent use of the dummy increases the association with SIDS.

If parents choose to use a dummy, practitioners should make them aware that:
\begin{itemize}
\item If the baby is breastfed that the use of a dummy can undermine breastfeeding especially before at least 6 weeks of age or until breastfeeding has become established.
\item It should be offered when settling the baby at every sleep episode (the protective factor appears to occur as the baby falls asleep).
\item If the dummy falls out of baby’s mouth once asleep, do not put back in.
\item If your baby does not seem to want the dummy, do not force them.
\item Do not coat the dummy in a sweet liquid.
\item Always clean and regularly replace dummies.
\end{itemize}

Try to wean your baby off their dummy by the age of one year.

\subsection{Are there recommendations for car seat or baby seat use?}
Some studies have shown that some infants, particularly pre-term infants or those with pre-existing 
health conditions are at risk of respiratory problems and/ or can experience slightly lower levels of
 oxygen in the blood if left for long periods in car seats. Also being left in a semi-upright position for 
ong periods may place strain on a baby’s developing spine.

Practitioners should advise parents that car seats are designed to keep babies safe whilst travelling 
so therefore should:

\begin{itemize}
\item Remove infants from car seats and place in Moses basket or cot.
\item Once inside the nursery or home, transfer baby into a cot or Moses basket and remove any outdoor clothing. 
\item When travelling on long journeys make regular stops and take baby out of the car seat for breaks. 
\end{itemize}


\subsection{How do I carry baby safely in a sling?}
Slings are carriers that allow an adult to carry an infant hands-free. The sling straps around
the adult’s neck, allowing the infant to lie in front of the adult, curved in a C-shape position.

If you choose to carry baby in a sling, at all times ensure that:

\begin{itemize}
\item baby’s airways are free at all times
\item you can see baby’s face
\end{itemize}

Babies can suffocate lying with a curved back with the chin resting on the chest or the face
pressed against the fabric of the sling or the wearer’s body. At particular risk from these
products are babies with a low birth weight, those that were born prematurely, or have
breathing issues such as a cold.

\subsection{Check list for safe sleeping}
\begin{enumerate}
\item Has baby been placed on the back to sleep?
\item Is baby sleeping in a safe Moses basket or cot away from hazards?
\item Does the cot meet British safety Standards for cots?
\item Is the mattress firm? 
\item Does the mattress fit the cot /Moses basket well?
\item Is the mattress clean and in good condition and flat (not titled or elevated)?
\item Is baby’s face and head uncovered?
\item Have any pillows, duvets, lamb’s wool, cot bumpers and soft toys been removed?
\item If using a baby sleeping bag, does it have a fitted neck, armholes or sleeves and no hood?
\item Remember never to sleep baby on a sofa, beanbag, waterbed or pillow?
\item If using blankets rather than a sleeping bag, has baby been placed
to sleep with feet touching the bottom of the cot /Moses basket with blankets securely tucked in.
\end{enumerate}


\input{includes/footer.tex}
