\documentclass[10pt]{article}

\usepackage{fullpage}
\usepackage{tabularx}
\usepackage{colortbl}
\usepackage{xcolor}
\usepackage{fontspec}
\usepackage{xcolor}
\usepackage{titlesec}
\defaultfontfeatures{Ligatures=TeX}
% Set sans serif font to Calibri
\setsansfont{Carlito}
% Set serifed font to Cambria
\setmainfont{Caladea}
% Define light and dark Microsoft blue colours
\definecolor{MSBlue}{rgb}{.204,.353,.541}
\definecolor{MSLightBlue}{rgb}{.31,.506,.741}
% Define a new fontfamily for the subsubsection font
% Don't use \fontspec directly to change the font
\newfontfamily\subsubsectionfont[Color=MSLightBlue]{Times New Roman}
% Set formats for each heading level
\titleformat*{\section}{\huge\bfseries\sffamily\color{MSBlue}}
\titleformat*{\subsection}{\Large\bfseries\sffamily\color{MSLightBlue}}
\titleformat*{\subsubsection}{\itshape\subsubsectionfont}

\setcounter{secnumdepth}{0}

\setlength{\parindent}{0pt}
\setlength{\parskip}{\baselineskip}%

\pagestyle{empty}

\begin{document}


\section{Safeguarding and Child Protection Policy and Procedures}

\subsection{Child Protection}

The purpose of this policy is to provide protection for all children in
my care --- this is my first responsibility and they are my priority.
Outlined is the procedure I will adopt if I have any cause for concern,
including how -- if necessary -- I will make a referral to the
children's safeguarding team (The Bridge Partnership, who may transfer me
to Duty \& Assessment Team or DAT).

I understand that child abuse can be physical, emotional, sexual,
neglectful or a mixture of these. I must notify Bridge/DAT and I am required
to record my concerns. I hold a copy of the relevant local procedures
and they are available on request.

Parents must notify me of any concerns they have about their child and
any accidents, incidents or injuries affecting the child, which will be
recorded.

If I notice:
\begin{itemize}[topsep=0pt]
\item
  Significant changes in the child's behaviour;
\item
  Unexpected bruising, marks or signs of possible abuse;
\item
  Comments that give me cause for concern;
\item
  Deterioration in general wellbeing that cause concern; or
\item
  Signs of neglect.
\end{itemize}

I will implement the procedure detailed below without delay to minimise
any risk to the child.

The following procedure will be undertaken in the event that I suspect a
child is at risk of harm:
\begin{itemize}[topsep=0pt]
\item
  Concerns will be discussed with parents, if appropriate.
\item
  Advice will be sought from The Bridge/DAT (0161 603 4500 -- or 0161 794 8888
  out of hours)
\item
  The ``Referral to Children's Services'' form will be completed
  (http://www.salford.gov.uk/secureupload.htm)
\item
  Once this referral is received, DAT will take responsibility and 
  decide the necessary steps to ensure the safety of the child.
\item
  In the event of an emergency, I will contact the police.
\end{itemize}

All information about the child is kept confidential, however in certain
situations, this information may be shared with The Bridge, DAT or the
police.

I complete annual safeguarding updates, in the past 12 months I have completed the 
online Safeguarding Children Refresher run by Virtual College and I have a copy of 
the ‘Working Together to Safeguard Children’ document available which is 
a government file for all agencies. I also have a copy of “What to do if you are 
worried a child is being abused” and I am part of an online group that shares safeguarding 
updates and articles of interest. I am required to complete external safeguarding 
training every two years and will book these through my local authority.

\subsection{Allegations of Abuse}

In the event of an allegation being made against the childminder -- or
any other adult living in the childminder's home -- The Local Authority
Designated Officer (LADO) will be informed immediately. All allegations
will be investigated independently by the LADO and will also be reported
to Ofsted.

In all instances, I will record:
\begin{itemize}[topsep=0pt]
\item
  The child's full name and address;
\item
  The date and time the record is being made;
\item
  Factual details of the concern, for example: bruising, what the
  child said and who was present;
\item
  Details of any previous concerns;
\item
  Details of any explanations from the parents; and
\item
  Any action taken, such as speaking to parents.
\end{itemize}

It is not my responsibility to attempt to investigate the situation
myself.

\subsection{Mobile Phones and Social Networking}

It is in the safety of the children in the childminder's care that the
childminder is in possession of a mobile phone at all times. This is to
be used to contact parents in the event of an emergency. The mobile
phone will only be used to take photos with the consent of parents and 
with the intention of sending these photos to the child's parents electronically
or for photographic evidence of development. These will be printed and stored in 
a child's Learning Journey as soon as possible, deleting the digital photo off of 
the mobile phone afterwards. For the safety of other 
children in my care, I request that parents do not use their mobile
phones within the setting.

I will not post any details of any children in my care on any
social network. I will discuss with parents individually if they wish 
for their child's image to be shared on my Facebook business page, Karlie's 
Childminding Services. With their consent I will post photos of their children
doing activities, but will not post photos of their faces. 

\subsection{Prevent Duty \& British Values}

I comply with the requirements of the Prevent Duty Guidance and its aim to
protect children from radicalisation, extremism and being drawn into terrorism.
I have shared information about the Prevent Strategy and my commitment to 
promote British Values with parents via the business' Facebook page and in person. 
If parents have any questions, please ask me.

\subsection{Peer on Peer Abuse}

I recognise that children and young people are capable of abusing their peers. Peer on peer abuse
relates to situations such as sexual exploitation, gang violence, financial abuse, coercive control and
exploitative relationships. I want all children to feel safe here and, as part of our commitment to keep
them safe, I regularly observe children’s interactions and aim to be approachable so they will speak to me
if they are concerned about any aspects of their relationships with others. Parents know they can contact
me at any mutually convenient time to discuss concerns children might raise at home. 

\subsection{Female Genital Mutilation (FGM)}

I have completed training in relation to FGM. I have been trained to look out for any signs that it might 
take place and how to report those to the relevant people, and that if I ever found out FGM had taken 
place in relation to any child under 18, I would have to report this directly to the police. 


\subsection{Useful Telephone Numbers}

\begin{table}[h]
  \begin{tabularx}{\textwidth}{lX}
    Ofsted & 0300 123 1231 \\
    LADO & 0161 603 4350 \\
    The Bridge/DAT & 0161 603 4500 (Out of hours 0161 794 8888) \\
    Starting Life Well & 0161 778 0384 \\ 
 \end{tabularx}
\end{table}

\begin{table}[h]
  \def\arraystretch{2.0}
  \begin{tabularx}{\textwidth}{|l|X|}
    \hline
    Childminder's name & \\
    \hline
    Childminder's signature &  \\
    \hline
    Date & \\
    \hline
  \end{tabularx}
\end{table}

\begin{table}[H]
  \def\arraystretch{2.0}
  \begin{tabularx}{\textwidth}{|X|X|X|X|}
    \hline
    Child's Name & Parent's Name & Signature & Date \\
    \hline
    ~ & ~ & ~ & \\
    \hline
    ~ & ~ & ~ & \\
    \hline
    ~ & ~ & ~ & \\
    \hline
    ~ & ~ & ~ & \\
    \hline
    ~ & ~ & ~ & \\
    \hline
  \end{tabularx}
\end{table}

Review Date: 31/08/2017

\input{includes/footer.tex}
