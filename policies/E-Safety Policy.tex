\documentclass[10pt]{article}

\usepackage{fullpage}
\usepackage{tabularx}
\usepackage{colortbl}
\usepackage{xcolor}
\usepackage{fontspec}
\usepackage{amsmath}
\usepackage{titlesec}
\defaultfontfeatures{Ligatures=TeX}
% Set sans serif font to Calibri
\setsansfont{Carlito}
% Set serifed font to Cambria
\setmainfont{Caladea}
% Define light and dark Microsoft blue colours
\definecolor{MSBlue}{rgb}{.204,.353,.541}
\definecolor{MSLightBlue}{rgb}{.31,.506,.741}
% Define a new fontfamily for the subsubsection font
% Don't use \fontspec directly to change the font
\newfontfamily\subsubsectionfont[Color=MSLightBlue]{Times New Roman}
% Set formats for each heading level
\titleformat*{\section}{\huge\bfseries\sffamily\color{MSBlue}}
\titleformat*{\subsection}{\Large\bfseries\sffamily\color{MSLightBlue}}
\titleformat*{\subsubsection}{\itshape\subsubsectionfont}

\usepackage{titlesec}
\titlespacing*{\subsection}{0pt}{1ex plus 1ex minus .2ex}{0ex}

\setcounter{secnumdepth}{0}

\setlength{\parindent}{0pt}
\setlength{\parskip}{\baselineskip}%

\pagestyle{empty}

\begin{document}

\section{E-Safety Policy}

The internet is an incredible resource for children to access, support
for their homework, chatting to friends etc. but it can also be a very
dangerous place for them.

They can be exposed to inappropriate material, harassment and
bullying, viruses and hackers and be conned into giving away financial
information. They can also be vulnerable to on-line grooming by
paedophiles.

As a Childminder I offer children the opportunity to use the computer
and the internet when supervised by me; however I have introduced a
range of procedures to ensure their safety.


\begin{itemize}

\item
Our computer is a tablet used only by myself and my young child and no
other files are kept on it.

\item
Children are always very carefully supervised when using the device.

\item
I do not permit the children to go onto chat rooms

\item
I talk to the children about the websites they are using

\item
I check the history on the computer regularly

\item
I discuss with the older children about the importance of keeping safe
on line, not forwarding on chain letters, not talking to people they
don’t know, not giving out personal information that could enable
people to identify them, to tell me if they are worried about anything
and to never arrange to meet anyone they have spoken to online.

\item
The children are not allowed to use the tablet camera or video
message.

\item
I have provided each of you with an online safety document from the
NSPCC as part of this update.

\end{itemize}

If you would rather your child was not allowed access to the internet
then please let me know.

I am also aware of the need to limit the time children spend on
computers and will develop strategies to ensure that they spend a
balance of time engaged in ICT and other activities.

If you have any concerns regarding this policy then please do not
hesitate to contact me.

If your child brings their own electrical items into my setting they
will be placed in to a cabinet and only used if they need to contact
someone or they need to check their telephones. If you need to contact
them, please contact me. This strict rule is in place to prevent the
children taking photographs of each other and sharing their location
using the internet to ensure all the children are safe.

Your child will not be able to take pictures, either on a camera or
phone, of any of the other children in my care.

Review Date: 13/02/2017

\end{document}

