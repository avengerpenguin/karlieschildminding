\documentclass[10pt]{article}

\usepackage{fullpage}
\usepackage{tabularx}
\usepackage{colortbl}
\usepackage{xcolor}
\usepackage{fontspec}
\usepackage{xcolor}
\usepackage{titlesec}
\defaultfontfeatures{Ligatures=TeX}
% Set sans serif font to Calibri
\setsansfont{Carlito}
% Set serifed font to Cambria
\setmainfont{Caladea}
% Define light and dark Microsoft blue colours
\definecolor{MSBlue}{rgb}{.204,.353,.541}
\definecolor{MSLightBlue}{rgb}{.31,.506,.741}
% Define a new fontfamily for the subsubsection font
% Don't use \fontspec directly to change the font
\newfontfamily\subsubsectionfont[Color=MSLightBlue]{Times New Roman}
% Set formats for each heading level
\titleformat*{\section}{\huge\bfseries\sffamily\color{MSBlue}}
\titleformat*{\subsection}{\Large\bfseries\sffamily\color{MSLightBlue}}
\titleformat*{\subsubsection}{\itshape\subsubsectionfont}

\setcounter{secnumdepth}{0}

\setlength{\parindent}{0pt}
\setlength{\parskip}{\baselineskip}%

\pagestyle{empty}

\begin{document}


\section{Equal Opportunities Policy}

It is important to recognise that every child is unique and an
individual. My setting aims to value and celebrate diversity and
recognise individual differences whether they be race, religion, gender,
age, physical ability or dietary beliefs. I will ensure that I treat all
children with equal concern and respect to meet their individual needs
and give them the opportunity to reach their full potential. I will
always help the children to feel good about themselves and others, by
celebrating the differences which make us unique.

I will offer a wide range of toys that reflect and celebrate diversity
and will celebrate and observe a variety of special dates from many
religions such as Eid, Christmas, Chinese New Year, Diwali etc. This
will be done through many ways such as cooking, baking, crafting and
artwork, parties etc.

I encourage the children in my care to learn more about their own
culture and to find out about the culture and religions of other
children through fun activities. This leads to them having a healthy
respect of each other's differences and to value everyone as an
individual.

Before a child registers at my setting an ``All About Me'' form will be
completed. This is to ensure that individual needs, of any type, are
discussed and recorded. These requirements will be adhered to at all
times. ~When necessary activities and the menu will be adapted
accordingly.

Each child is observed and ``Next Steps'' are planned for them to
develop and learn. Activities will be organised specifically for their
developmental stage.

If a child has any additional needs or disabilities that require me to
have further training this will be completed as soon as is realistically
possible. I will work alongside parents/carers and outside agencies to
ensure that each child's needs are met at all times. I encourage
parents/carers to share with us any festivals or special occasions which
may enhance the children's learning and understanding.

Date: 07/06/2014

\input{includes/footer.tex}

