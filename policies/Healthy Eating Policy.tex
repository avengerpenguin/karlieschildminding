\documentclass[10pt]{article}

\usepackage{fullpage}
\usepackage{tabularx}
\usepackage{colortbl}
\usepackage{xcolor}
\usepackage{fontspec}
\usepackage{xcolor}
\usepackage{titlesec}
\defaultfontfeatures{Ligatures=TeX}
% Set sans serif font to Calibri
\setsansfont{Carlito}
% Set serifed font to Cambria
\setmainfont{Caladea}
% Define light and dark Microsoft blue colours
\definecolor{MSBlue}{rgb}{.204,.353,.541}
\definecolor{MSLightBlue}{rgb}{.31,.506,.741}
% Define a new fontfamily for the subsubsection font
% Don't use \fontspec directly to change the font
\newfontfamily\subsubsectionfont[Color=MSLightBlue]{Times New Roman}
% Set formats for each heading level
\titleformat*{\section}{\huge\bfseries\sffamily\color{MSBlue}}
\titleformat*{\subsection}{\Large\bfseries\sffamily\color{MSLightBlue}}
\titleformat*{\subsubsection}{\itshape\subsubsectionfont}

\setcounter{secnumdepth}{0}

\setlength{\parindent}{0pt}
\setlength{\parskip}{\baselineskip}%

\pagestyle{empty}

\begin{document}
\section{Healthy Eating Policy}

I believe it is essential to provide children with positive, healthy
eating experiences in order to promote their well being. I also
encourage children to look at the long-term effects of a healthy and
balanced diet. I respect the different dietary, cultural and health
needs of all the children.

I will provide a healthy menu which meets the nutritional requirements
and dietary needs of children as they grow. These menus are available to
view at any time and will be reviewed seasonally.

I have a vegetarian household and as such my menus will be suitable for
Ovo-Lacto Vegetarians. Should your child have any specific dietary need
or allergy, please let me know and I will accommodate this.
Alternatively, you can provide meals for your child yourself, but these
must be in line with this policy.

My menu is designed to promote family dining in the evening. For this
reason the larger meal is served at lunch time, with a smaller portion
being served at teatime, to ensure that the children are still able to
eat dinner with their families when they get home. Parents can discuss
with me if they require their child to be fed dinner whilst in my care,
and under certain circumstances exceptions can be made.

I encourage children to choose healthy options and to experiment by
trying new foods from other cultures. Older children are encouraged to
help in the preparation of food and meals.

I will record what your child has eaten and approximate amounts in the
daily diary. If you have any concerns regarding diet/menu/quantity
please do not hesitate to discuss it with me.

I am happy to support you if you are weaning your baby. I have a blender
and am willing to make puréed dishes if required.

Children are offered water, milk or occasionally fruit juices. I do not
allow children to have fizzy drinks. Water is available at all times for
the children to access independently.

I have completed the Level two food safety and hygiene for catering
training, and as a result am aware of the correct procedures for
handling, storing and serving food.

Date: 07/06/2014

~
\end{document}
