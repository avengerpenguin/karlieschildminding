\documentclass[10pt]{article}

\usepackage{fullpage}
\usepackage{tabularx}
\usepackage{colortbl}
\usepackage{xcolor}
\usepackage{fontspec}
\usepackage{xcolor}
\usepackage{titlesec}
\defaultfontfeatures{Ligatures=TeX}
% Set sans serif font to Calibri
\setsansfont{Carlito}
% Set serifed font to Cambria
\setmainfont{Caladea}
% Define light and dark Microsoft blue colours
\definecolor{MSBlue}{rgb}{.204,.353,.541}
\definecolor{MSLightBlue}{rgb}{.31,.506,.741}
% Define a new fontfamily for the subsubsection font
% Don't use \fontspec directly to change the font
\newfontfamily\subsubsectionfont[Color=MSLightBlue]{Times New Roman}
% Set formats for each heading level
\titleformat*{\section}{\huge\bfseries\sffamily\color{MSBlue}}
\titleformat*{\subsection}{\Large\bfseries\sffamily\color{MSLightBlue}}
\titleformat*{\subsubsection}{\itshape\subsubsectionfont}

\setcounter{secnumdepth}{0}

\setlength{\parindent}{0pt}
\setlength{\parskip}{\baselineskip}%

\pagestyle{empty}

\begin{document}
\section{Medication Policy}

Medicine cannot be administered without written permission from the
parent/carer using a medicine form provided by me. The form will have
information stating the dosage required (using the guidance on the
medicine bottle), the times the medicine needs to be given, and the
reason why.

This form will be filled in by the parent/carer daily until the medicine
is no longer required. I will then fill in the rest of the form when
medicine is given. The parent/carer is then required to sign the form to
acknowledge the administration of medicine when they collect their
child.

If your child needs to take medication prescribed by a doctor, please
discuss this with me. In some cases a child on antibiotics may be asked
not to attend for 2-3 days in case they react to the medication and to
prevent the spread of an infection to others.

I am happy to give your child non-prescribed medication, such as cough
mixture, Calpol, teething gel etc., but only if you have signed a
parental permission form for me to do so and provide the medication
yourself. This must be clearly labelled with the child's name.

It is vital that you inform me of any medication you may have given your
child before they arrive into my care. I need to know what medicine they
have had, the dose and time given.

I will ensure that all medication given to me will be stored correctly
and I will check that it is still within its expiry date. Children's
medicines will be stored in a sealed box either on the top shelf of the
fridge or a cupboard. My own medicines will be stored in a cupboard, far
out of children's reach.

The parent/carer are required to fill in a care plan before the child
attends the setting, stating any relevant medical or dietary
requirements, which will be adhered to at all times. If necessary I may
need to receive training on your child's medical needs, to ensure their
needs are met.

Date: 07/06/2014

~
\end{document}
